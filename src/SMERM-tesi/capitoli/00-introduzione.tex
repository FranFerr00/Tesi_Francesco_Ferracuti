% !TEX TS-program = pdflatex
% !TEX root = ../tesi.tex

%************************************************
\pdfbookmark{Introduzione}{Introduzione}
\chapter*{Introduzione}
\label{chp:introduzione}
%************************************************

La tesi espone il lavoro, durato circa un anno, di studio e ricerca
sull'aumentazione del flauto. Sono diversi i prototipi che si sono sviluppati e
succeduti nel corso dell’elaborazione di questo percorso per arrivare alla
definizione del prototipo qui presentato con il nome di \emph{Flauto Bicamerale}
[CAP. \ref{chp:nascita}].
 
Il primo passo è stato quello di prendere in esame alcune esperienze precedenti
di strumenti aumentati basati sul controllo di sistemi di feedback e di sistemi
risonanti [CAP. \ref{chp:strumentiaumentati}].

Parallelamente al succedersi dei vari prototipi, è stato approfondito l’ambito
teorico che ha generato ulteriori stimoli  di ricerca utili  alla definizione
del progetto. [\ref{chp:discreto}].

Dal punto di vista acustico e musicale la ricerca nasce dalla considerazione che
molta letteratura \emph{live electronics} è determinata dalla catena che consiste
in uno strumento che suona, un computer che rielabora tramite processi e la
diffusione del suono elaborato tramite altoparlanti.

Inizialmente l’idea che ha determinato l’elaborazione di questo strumento
aumentato consisteva nel ribaltare tale catena, abolendo il doppio ascolto che
si percepisce quando si è in presenza un brano live [\ref{chp:ricerca}].

Durante il processo di elaborazione dello strumento aumentato, però, questa idea
iniziale si è ampliata inglobando la ricerca, tramite il computer, di nuovi
timbri e tecniche da poter produrre dentro lo strumento. Si è cercato così di
generare delle nuove sonorità, impossibili da ottenere con il suono classico
dello strumento, in questo caso di un flauto, o con la sola sintesi al computer.

Lo strumento tradizionale risulta discretizzato e diviso in frequenze ben
distinte secondo una prassi linguistica musicale che permette la facile
trascrizione della musica su carta usando una logica matematica facilmente
trasmissibile. La partitura in questo modo contiene tutte le informazioni
necessarie e sufficienti alla riproposizione del brano.

Se possiamo definire “astratta” la notazione classica questo prototipo produce
usa “concretizzazione” dello strumento. Le sonorità ottenute, infatti, non sono
trascrivibili in partiture notazionali. Il reciproco ascolto e la progressiva
sintonia tra i il flautista e l’interprete \emph{live electronics} è parte determinante
della composizione.

Questa alternarsi di indicazioni e scelte da parte dei due interpreti è
paragonabile alla perduta struttura bicamerale della mente umana. Tale struttura
è stata studiata raffrontandola con i contesti comunicativi del linguaggio della
percezione visiva e della musica.
