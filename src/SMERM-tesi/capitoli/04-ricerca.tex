% !TEX TS-program = pdflatex
% !TEX root = ../tesi.tex

%************************************************
\chapter{LA RICERCA MUSICALE}
\label{chp:ricerca}
%************************************************

Una volta messo a punto  lo strumento sono iniziati i processi di ricerca  per capirne le potenzialità musicali.

Già durante il processo di  sviluppo del sistema altoparlante-flauto, anche se appena accennate,  sono affiorate alcune potenzialità sonore molto interessanti:
Usando  basse frequenze si formano frullati o tremoli, delicati e controllati, non realizzabili con un flauto tradizionale
 Nella banda critica si accentuano i battimenti dal momento che il suono entra direttamente dentro lo strumento.
Queste potenzialità sono state approfondite con lo strumento ultimato, specificando quattro modalità operative:
Con l’uso di una sinusoide pura tra i 1hz e i 100 hz si ricava un tremolo controllato dentro il flauto. Le diverse  frequenze del tremolo variano  in base alla frequenza della sinusoide,
Alla frequenza della sinusoide del punto 1 si somma  una altra sinusoide, anche essa tra i 1 hz e i 100 hz , formando così una modulazione di frequenza.  Con questa modalità risulta possibile sviluppare spettri più ricchi creano leggere variazioni ritmiche  nel tremolo .
Con l’uso di un rumore bianco, filtrato con un bandpass di trentaduesimo ordine, e  con un q variabile ma  molto stretto,si ricava, con frequenze molto gravi, un leggero senso di indeterminabilità . Viene così prodotta  una ritmicità che pur controllata ha una tendenza verso l’aleatorietà.          Quando questa modalità viene  spostata  nelle

bassissime frequenze aggiunge alle precedenti sonorità simili a soffi date dallo spingere i filtri al limite della rottura.
Per quanto riguarda la zona della  banda critica, una modulazione ad anello del suono del flauto catturato con un microfono, moltiplicato   con una sinusoide tra i 20hz e i 50 hz, produce  dei battimenti su ogni formante del flauto creando spettri ricchissimi.

Il brano, ancora in fase di sviluppo, in alcune zone, denominate “voci” necessita dell’ascolto vicendevole e della simultanea collaborazione  dei due interpreti, il flautista ed il live electronics, per la ricerca della massima sintonia possibile in presenza di continue variazioni determinate da scelte alternate dei due interpreti.
Ognuna di queste scelte è determinata da tre possibilità, espresse in partitura,  che l’interprete è libero di optare. L’altro interprete, ascoltata e capita la scelta, agisce conseguentemente alle indicazioni in partitura relative alla scelta operata dal primo.
Le “voci” sono un “tempo sospeso”, non riportato in partitura e non determinabile a priori.

Per quanto riguarda l’elettronica  in partitura sono indicate le modalità di operazione:
Sine = sinusoide
FM = modulazione di frequenza


N = noise
RM  = modulazone di ampiezza
Di seguito a queste modalità di operazione è indicato il range frequenziale nel quale operare.
Per quanto riguarda la modulazione di frequenza il primo range riguarda la modulante e il secondo della portante.
Nel noise il primo parametro indica la frequenza di operazione e il secondo il q.
I vari gesti di l.e. sono completati con delle indicazioni grafiche che descrivono prima i percorsi frequenziali e poi gli andamenti dinamici dei percorsi di ampiezza.
Ogni “voce” ha al suo interno tre possibilità di scelta di gesto sonoro.
Il verso di una freccia indica quale è l’ interprete  che deve operare la scelta.
La scelta operata produce una modalità sonora che deve essere riconosciuta dall’altro interprete in modo da potersi sintonizzare con essa.
In questo “tempo sospeso” nessuno dei due interpreti smette di suonare  in quanto il primo continua nella sua modalità ed il secondo sperimenta la sincronizzazione.
Non esistono  dei tempi definiti per la  durata di ogni “voce”.
Infatti l’istante di ritorno alla partitura non risulta prevedibile  in quanto  sarà in completa gestione dell’interprete “comandante” nella “voce”suonata.

In questo modo si formano due interpreti bicamerali.
L’interprete comandante diventa in questo modo parte emotiva e “allucinazione sonora” del secondo interprete. Infatti il secondo interprete riceve l’ordine del gesto sonoro dal primo e tramite ascolto e processi logici deve eseguire  il conseguente gesto sonoro.
